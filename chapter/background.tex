\chapter{Theoretical Background}

In this chapter we will present a brief introduction to neural networks and then describe the underlying concepts of the Convolutional Neural Network (CNN).
We will focus on presenting the main concepts that will be used in the development of pose estimation algorithms.

\section{Neural network}

In the Neural Networks chapter, we will talk in the first part about a brief history, presenting the three periods through which artificial intelligence has passed.
Then we will address the biological part that is found in neural networks.
In the next section we will discuss what is a neural network , how many types of such networks are there? What is a neuron? But an activation function?
Finally, we will discuss learning such a neural network with its specific algorithms.

\subsection{History}
Artificial intelligence takes us to think of some SF films, 
but it still has a long way to go, for now these intelligent algorithms can
only get to the level of intelligence of an insect, 
they work better in certain exact tasks as imagine detectie  and not in general like a brain.

But let's not forget that this domain has been built around the dream to overcome human intelligence.
The essential question is whether such a system can be implemented on a computer?

The domain of psychology was the first to have had the artificial intelligence applicability, 
the most famous Turing test that appeared in 1950. 
It involves a conversation of a person with a computer and another person and he has to guess who is the person. \cite{historyofneuronalnetwork}

Three great periods are in the history of artificial intelligence. 
In the first period, only after the Second World War. The first programs that implement various smart algorithms to solve puzzles. \cite{historyofneuronalnetwork}

An important algorithm in this field is Samuels' game, it was quite simple to implement. They save certain winning positions throughout the game.\cite{historyofneuronalnetwork}
The first period kept until 1965, but no algorithm has led to major changes in people's lives. \cite{historyofneuronalnetwork}
\subsection{Biological}
\subsection{Generalitati}
\subsection{Back propagation}

\section{Convolutional Neural Network}
\subsection{What is differente?}


% \section{The purpose of the application}