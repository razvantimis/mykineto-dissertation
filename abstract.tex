\begin{abstract}
\begin{comment}

Cei mai multi pacienti care necesita recuperare medicala pe termen lung, asta insemnand in medie 2 sau 3 ani, renunta pe parcursul perioadei de recuperare deoarece nu observa rezultate chiar daca acestea exista, ele nu sunt evidente cu ochiul liber.
Mai ingrijorator este că până la 70\% din pacienți renunță la fizioterapie, deoarece nu pot vedea rezultate imediate \cite{7FactsInPhysicalTherapy}.
Solutia nostra este o aplicatie prin care pacientul va putea sa isi inregistreze sesiunile de recuperare iar in real time va primi feedback constant legat de Range of Motion si numarul de repetati facute corect. In urma acestor date vom genera raporte prizind progresul pacientului, astfel vom incurajeze pacienti sa isi continua recuperarea medicala pe tot parcursul ei. 
În fizioterapie, gama de urmărire a mișcării (ROM) este o abordare standard pentru măsurarea progresului în terapia pacienților.
Pentru obtinerea de Range of Motion vom folosi algoritmi intelegenti de detectare a posturi in timp real.

\end{comment}

\par The goal of this paper is to obtain reports on patient progress based on intelligent pose estimation  algorithms motivate the patient to continue medical recovery.
\par Most patients require long-term medical recovery, which means an average of 2 or 3 years, give up during the recovery period because they do not see results even if they exist, they are not obvious to the naked eye.

\par More worrying is that up to 70\% of patients give up physiotherapy because they can not see immediate results. \cite{7FactsInPhysicalTherapy}.

\par Our solution is an application called My Kineto through which the patient will be able to record their therapy sessions and will receive constant feedback in real time about Range of Motion. Using this data these data, we will generate reports on patient progress.





\end{abstract}